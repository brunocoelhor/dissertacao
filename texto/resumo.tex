%Resumo em Portugues (no maximo 500 palavras)
\begin{abstract}
Sistemas baseados em Inteligência Artificial Distribuída (IAD) são utilizados para solucionar problemas computacionais que os sistemas tradicionais não são capazes de solucionar ou até como uma nova abordagem para soluções já existentes. Os Sistemas Multiagentes (SMA) fazem parte de um dos ramos da IAD, e além das propriedades comuns da IAD como ser um sistema assíncrono e distribuído os agentes ainda possuem propriedades como autonomia, reatividade, pro-atividade e habilidade social o que torna muito difícil prever comportamentos nestes sistemas. 

Para limitar o comportamento em SMA modelos organizacionais, como o Moise, podem ser empregados para especificar u sistema a partir de um modelo de organização. Estes modelos  estruturam os agentes em grupos, onde os membros destes grupos tem papéis a desempenhar e também restrições a obedecer. 

Mesmo com este nível de controle sobre os SMA, comportamentos inesperados podem surgir. Para garantir que comportamentos imprevistos não prejudiquem o funcionamento do sistema, e assim assegurar a qualidade do software, técnicas de teste de software devem ser empregadas como uma das estratégias. 

Assim esta proposta de dissertação tem por objetivo propor um método para avaliar a testabilidade em SMA que empregam o modelo de organização Moise, utilizando Rede de Petri (RP) como ferramenta de descrição e análise. O método é baseado em uma técnica de avaliação de testabilidade para agentes BDI, que deve ser mapeado para Redes de Petri, e daí para o modelo MOISE. O resultado indica o número de casos de uso necessários para garantir a correção do sistema.
\end{abstract}